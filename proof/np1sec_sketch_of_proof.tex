\documentclass{article}
\usepackage[english]{babel}

%%%%%%%%%% Start TeXmacs macros
\newcommand{\tmstrong}[1]{\textbf{#1}}
%%%%%%%%%% End TeXmacs macros

\begin{document}

\title{Sketch of Security Proof for (n+1)Sec Protocol}

\maketitle

The (n+1)Sec protocol is composed of following sub protocol:
\begin{enumerate}
  1. {\tmstrong{TDH}}: Triple DH deniable Authentication
  
  2. {\tmstrong{FAGKE}}: Flexible Authenticated Group Key Exchange protocol
  presented in {\cite{AMP10}}
  
  3. {\tmstrong{SecCom}}: Secure (authenticated confidential) Send and
  Receive.
  
  4. {\tmstrong{TCA}}: Transcript Consistency Assurance.
\end{enumerate}
The threat model for each of these protocol is described in Section VI. The
security of FAGKE is proven in the presented threat model. The SecComm
consists of convential ``sign'' and ``encrypt'' functions and its security has
been studied as a subprotocol to various protocols. We are not aware of any
existing proof for TDH and TCA subprotocol.

The sketch of the proof goes as follows, in Section \ref{sect-tdh-sec} and
Section \ref{sect-tca-sec} we give convential formal proof of the security
properties of TDH and TCA respectively. In Section \ref{sect-np1sec-pclize} we
reforumlates the proves of all four protocols in Protocol Composition Logic
(PCL). In Section \ref{sect-comp-sec}, we proof the security of (n+1)sec by
proving the relative security of above sub prorotocol in relation to each
other:
\begin{enumerate}
  \item $Q_1$ as Parallel composition of TDH and FAGKE.
  
  \item Sequential composition of $Q_1$ and SecCom.
  
  \item  Parallel compostion of SecCom and TCA.
\end{enumerate}

\section{Security of Triple Deffie-Hellman Authentication}

\label{sect-tdh-sec}

\section{Security of Transcript Consistency Assurance}

\label{sect-tca-sec}

\section{(n+1)Sec components in PCL Langugae}

\label{sect-np1sec-pclize}

\section{Security of composed sub protocols}

\label{sect-comp-sec}

\end{document}
