\documentclass{article}
\usepackage[english]{babel}
\usepackage{amsmath,latexsym}

%%%%%%%%%% Start TeXmacs macros
\newcommand{\nocomma}{}
\newcommand{\tmop}[1]{\ensuremath{\operatorname{#1}}}
\newcommand{\tmstrong}[1]{\textbf{#1}}
\newenvironment{proof}{\noindent\textbf{Proof\ }}{\hspace*{\fill}$\Box$\medskip}
\newtheorem{definition}{Definition}
\newtheorem{theorem}{Theorem}
\providecommand{\xequal}[2][]{\mathop{=}\limits_{#1}^{#2}}
%%%%%%%%%% End TeXmacs macros

\begin{document}

\title{Sketch of Security Proof for (n+1)Sec Protocol}

\maketitle

The (n+1)Sec protocol is composed of following sub protocol:
\begin{enumerate}
  1. {\tmstrong{TDH}}: Triple DH deniable Authentication
  
  2. {\tmstrong{FAGKE}}: Flexible Authenticated Group Key Exchange protocol
  presented in {\cite{AMP10}}
  
  3. {\tmstrong{SecCom}}: Secure (authenticated confidential) Send and
  Receive.
  
  4. {\tmstrong{TCA}}: Transcript Consistency Assurance.
\end{enumerate}
The threat model for each of these protocol is described in Section VI. The
security of FAGKE is proven in the presented threat model. The SecComm
consists of convential ``sign'' and ``encrypt'' functions and its security has
been studied as a subprotocol to various protocols. We are not aware of any
existing proof for TDH and TCA subprotocol.

The sketch of the proof goes as follows, in Section \ref{sect-tdh-sec} and
Section \ref{sect-tca-sec} we give convential formal proof of the security
properties of TDH and TCA respectively. In Section \ref{sect-np1sec-pclize} we
reforumlates the proves of all four protocols in Protocol Composition Logic
(PCL). In Section \ref{sect-comp-sec}, we proof the security of (n+1)sec by
proving the relative security of above sub prorotocol in relation to each
other:
\begin{enumerate}
  \item $Q_1$ as Parallel composition of TDH and FAGKE.
  
  \item Sequential composition of $Q_1$ and SecCom.
  
  \item  Parallel compostion of SecCom and TCA.
\end{enumerate}

\section{Security of Triple Diffie-Hellman Authentication}

\subsection{The Triple Diffie-Hellman Protocol}

\begin{table}[tbh]
  \begin{tabular}{lll}
    Round 1 & $A \rightarrow B :'' A'', g^a$ & $B \rightarrow A :'' B'',
    g^b$\\
    Key Computation & $k \leftarrow H ((g^b)^A | (g^B)^a | (g^b)^a)$ & $k
    \leftarrow H ((g^A)^b | (g^a)^B | (g^a)^b)$\\
    Round 2 & $\tmop{Enc}_k (H (k, A))$ & $\tmop{Enc}_k (H (k, B))$
  \end{tabular}
  \caption{}
\end{table}Assuming that $A$ and $B$ are represeneted by long term public key
$g^A$ and $g^B$ respectively:

\subsection{The deniablity of TDH}

\label{sect-tdh-sec} We will prove a parallel to Theorem 4 {\cite{GKR06}}
which proves the deniability of SKEME. We use the notation which are
introduced in Section \ref{sect-deniabl-adv}. Following the same notation:

\begin{definition}
  By $\tmop{Adv}_{\tmop{deny}}^{\ast}$ we represent the party which represent
  the interaction of the Simulator $\tmop{Sim}$ with the adverasy. In other
  word, $\tmop{Adv}^{\ast}_{\tmop{deny}}$ has access to all information which
  $\tmop{Adv}_{\tmop{deny}}$ possess.
\end{definition}

\begin{theorem}
  If Computational Diffie-Hellman (CDH) is interactable then Triple DH
  Algorithm is deniable.
\end{theorem}

\begin{proof}
  We build $\tmop{Sim}_{}$ which interacts with $\tmop{Adv}_{\tmop{deny}}$. We
  show that if $\mathcal{J}$ is able to distinguish
  $\tmop{Trans}_{\tmop{Sim}}$ from $\tmop{Trans}_{\tmop{Real}}$, ze should be
  able to solve CDH as well.
  
  Intuitively, when $\mathcal{A}_{\tmop{deny}}$ sends $g^a$ to $\mathcal{}
  \mathcal{S}_{\tmop{deny}}$, $\mathcal{} \mathcal{S}_{\tmop{deny}}$ inquire
  $\mathcal{A}_{\tmop{deny}}$ for $a$, in this way $\mathcal{}
  \mathcal{S}_{\tmop{deny}}$ also can compute the same key $k$ by asking
  $\mathcal{A}_{\tmop{deny}}^{\ast}$. If $\mathcal{A}_{\tmop{deny}}$ has
  chosen $g^a \in \tmop{Tr} (B)$ or just chosen a random element of the group
  without knowing its DLP, then $\mathcal{S}_{\tmop{deny}}$ will choose a
  random exponent $a'$ and computes the key $k$ based on that and computes the
  confirmation value using $k$. Due to hardship of CDH this value is
  indistinguishable from a $k$ generated by $B$
  
  Now we suppose that the TDH is not deniable and we build a solver for CDH.
  First we note that if $\mathcal{A}_{\tmop{deny}}$ engages in an honest
  interaction with $B$ there is no way that $\mathcal{J}$ can distinguish
  between the $T (\mathcal{A}_{\tmop{deny}} (\tmop{Aux}))$ and $T
  (\mathcal{S}_{\tmop{deny}} (\tmop{Aux}))$. As $\mathcal{A}_{\tmop{deny}}$ is
  able to generate the very exact transcript without help of $B$. Therefore,
  logically, the only possibility for $\mathcal{J}$ to distinguish $T
  (\mathcal{A}_{\tmop{deny}} (\tmop{Aux}))$ and $T (\mathcal{S}_{\tmop{deny}}
  (\tmop{Aux}))$ is when $\mathcal{A}_{\tmop{deny}}$ present $\mathcal{J}$
  with a transcript that $\mathcal{A}_{\tmop{deny}}$ is not able to generate
  zirself. The only variable that $\mathcal{A}_{\tmop{deny}}$ has control over
  in the course of the exchange is $g^a$ and therefore the only way
  $\mathcal{A}_{\tmop{deny}}$ is able to claim that ze were unable to generate
  the geneuine \ $T (\mathcal{A}_{\tmop{deny}} (\tmop{Aux}))$ is by submiting
  $g^a$ which zirself does not know about its $a$ exponent.
  
  In such case, assuming the undeniability of TDH we have an $\varepsilon$
  such that
  \[  \max_{\tmop{all} \mathcal{J}} |2 \Pr (\tmop{Output} (\mathcal{J},
     \tmop{Aux}) = b) - 1| > \varepsilon \]
  The solver $\mathcal{A}_{\tmop{CDH}}$ receives a triple $(g, g^a, g^b)$ and
  should compute $g^{a b}$. To that end, assuming long term identiy $g^A$ for
  some $\mathcal{A}_{\tmop{deny}}$, ze engages ,in a TDH key exchange with a
  hypothetical automated party $\mathcal{A}^{\ast}$ with long term private key
  $B$ who generates $g^b$ as the ephemeral key as well.
  $\mathcal{A}_{\tmop{CDH}}$, then toss a coin and based on the result it
  either choose a random $a'$ and compute $g' = g^{a'}$ or set $g' = g^a,$then
  ze submits $h_0 = H (g^{b A} \nocomma, g'^B, g^{b a'})$ along side with
  $(g^B, g^b)$ to the $\mathcal{J}$ as a proof of engagement with
  $\mathcal{A}^{\ast}$. Due to undeniability assumption
  \[ \tmop{Output} (\mathcal{J}, \tmop{Aux}) (h_0, (A, g^a, B, g^b)) = b \]
  with significant probablity as means $\mathcal{J}$ is able to distinguish $T
  (\mathcal{A}_{\tmop{deny}} (\tmop{Aux}))$ and $T (\mathcal{S}_{\tmop{deny}}
  (\tmop{Aux}))$ with high probablity. Therefore $\mathcal{J}$ is able to
  decide if:
  \[ h_0 \xequal{?} \nocomma H (g^{b A} \nocomma, (g^a)^B, (g^a)^b) \]
  Because $H$ is a random oracle the only way that the judge is able to
  distinguish the second value from the real value is to have knowledge about
  the exact pre-image: $g^{b A} \nocomma, (g^a)^B, (g^a)^b$. Using the
  information in the transcript $\mathcal{J}$ can compute $g^{b A} \nocomma,
  (g^a)^B$, but still has to compute $g^{\tmop{ab}}$ using $g^a$ and $g^b$
  with high probablity without knowing $a$ or $b$, at this point
  $\mathcal{A}_{\tmop{CDH}}$ is publishing the value of $g^{a b}$.
  
  \ 
\end{proof}

\section{Security of Transcript Consistency Assurance}

\label{sect-tca-sec}

\section{(n+1)Sec components in PCL Langugae}

\label{sect-np1sec-pclize}

\section{Security of composed sub protocols}

\label{sect-comp-sec}

\end{document}
